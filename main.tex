\documentclass[12pt, a4paper]{article}

\usepackage[utf8]{inputenc}
\usepackage[brazil]{babel}

\usepackage[centertags]{amsmath}
\usepackage[svgnames]{xcolor}
\usepackage[section]{placeins}

\usepackage[lmargin=3cm, rmargin=3cm, tmargin=3cm, bmargin=3cm]{geometry}

\usepackage{amsmath, amsfonts, amssymb, graphicx, enumerate, verbatim, hyperref, color, lipsum, ifxetex, ifluatex, etoolbox, tikz, framed, libertine, listings, caption, subcaption, wrapfig, sectsty, pdfcomment}

\newcommand{\HRule}{\rule{\linewidth}{1pt}}

\sectionfont{\LARGE}
\subsectionfont{\Large}
\subsubsectionfont{\large}

/home/tonussi/Git/maquinas-jflap/maquinas/color.tex

\hypersetup{colorlinks,citecolor=MidnightBlue, filecolor=DarkSlateBlue, linkcolor=LightSlateBlue, urlcolor=SkyBlue}

\lstset{language=Java, basicstyle=\ttfamily, keywordstyle=\color{javapurple}\bfseries, stringstyle=\color{javared}, commentstyle=\color{javagreen}, morecomment=[s][\color{javadocblue}]{/**}{*/}, numbers=left, numberstyle=\tiny\color{black}, stepnumber=1, numbersep=10pt, tabsize=2, showspaces=false, showstringspaces=false}


% conditional for xetex or luatex
\newif\ifxetexorluatex
\ifxetex
  \xetexorluatextrue
\else
  \ifluatex
    \xetexorluatextrue
  \else
    \xetexorluatexfalse
  \fi
\fi
%
\ifxetexorluatex%
  \usepackage{fontspec}
  \usepackage{libertine} % or use \setmainfont to choose any font on your system
  \newfontfamily\quotefont[Ligatures=TeX]{Linux Libertine O} % selects Libertine as the quote font
\else
  \usepackage[utf8]{inputenc}
  \usepackage[T1]{fontenc}
  \usepackage{libertine} % or any other font package
  \newcommand*\quotefont{\fontfamily{LinuxLibertineT-LF}} % selects Libertine as the quote font
\fi

\newcommand*\quotesize{60} % if quote size changes, need a way to make shifts relative
% Make commands for the quotes
\newcommand*{\openquote}
   {\tikz[remember picture,overlay,xshift=-4ex,yshift=-2.5ex]
   \node (OQ) {\quotefont\fontsize{\quotesize}{\quotesize}\selectfont``};\kern0pt}

\newcommand*{\closequote}[1]
  {\tikz[remember picture,overlay,xshift=4ex,yshift={#1}]
   \node (CQ) {\quotefont\fontsize{\quotesize}{\quotesize}\selectfont''};}

% select a colour for the shading
\colorlet{shadecolor}{Azure}

\newcommand*\shadedauthorformat{\emph} % define format for the author argument

% Now a command to allow left, right and centre alignment of the author
\newcommand*\authoralign[1] {
  \if#1l
    \def\authorfill{}\def\quotefill{\hfill}
  \else
    \if#1r
      \def\authorfill{\hfill}\def\quotefill{}
    \else
      \if#1c
        \gdef\authorfill{\hfill}\def\quotefill{\hfill}
      \else\typeout{Invalid option}
      \fi
    \fi
  \fi
}


% wrap everything in its own environment which takes one argument (author) and one optional argument
% specifying the alignment [l, r or c]
\newenvironment{shadequote}[2][l]%
{\authoralign{#1}
\ifblank{#2}
   {\def\shadequoteauthor{}\def\yshift{-2ex}\def\quotefill{\hfill}}
   {\def\shadequoteauthor{\par\authorfill\shadedauthorformat{#2}}\def\yshift{2ex}}
\begin{snugshade}\begin{quote}\openquote}
{\shadequoteauthor\quotefill\closequote{\yshift}\end{quote}\end{snugshade}}


\begin{document}
\begin{titlepage}
\begin{center}

\textsc{\large Universidade Federal de Santa Catarina}\\[1cm]

\includegraphics[width=0.3\textwidth]{ufsc}\\[1.5cm]

\textsc{\LARGE \bfseries Teoria da Computação \\ [0.8cm]}

\textsc{\LARGE \bfseries Problema da Correspondência de Post \\ [3cm]}


\begin{Large}
\textbf{Estudante}:
\begin{tabular}{|c}
Luc$\lambda$s Tonussi 12106577\\
\end{tabular} \\[0.5cm]
\end{Large}

\vfill

\begin{Large}
\textbf{Professora}:
\begin{tabular}{|c}
Jerusa Marchi \\
\end{tabular} \\[0.5cm]
\end{Large}

\vfill


\textbf{\today}

\end{center}
\end{titlepage}


\tableofcontents
\clearpage
\listoffigures
\clearpage


% Requisito mínimo: aplicação funcionando.
% As seguintes questões serão analisadas:

\section{Máquinas}

$$ L(M)= \lbrace w \vert w \in \Sigma =  {\lbrace a,b,c \rbrace}^* \text{ e $w$ possui número par de $b$'s e o número de $a$'s + $c$'s é ímpar } \rbrace$$

$$ L(M)= \lbrace w \vert w \in \Sigma =  {\lbrace a,b,c \rbrace}^* \text{ e } w = {{a}^m}{{b}^n}{{c}^k} \text{ para } m \leqslant n  \leqslant 2m \text{ e } k \geqslant 0 \rbrace $$

$$ L(M)= \lbrace w \# w \vert w \in  {\lbrace 0,1 \rbrace}^* \rbrace $$

$$ L(M)= \lbrace ww \vert w \in  {\lbrace 0,1 \rbrace}^* \rbrace \textbf{ (MT-n) } $$


% \begin{figure}[ht]
% \centering
% \includegraphics[width=0.7\textwidth]{logica}
% \caption{Logica}
% \end{figure}




\end{document}
