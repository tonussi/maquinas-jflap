\documentclass[12pt, a4paper]{article}

\usepackage[utf8]{inputenc}
\usepackage[brazil]{babel}

\usepackage[centertags]{amsmath}
\usepackage[svgnames]{xcolor}
\usepackage[section]{placeins}

\usepackage[lmargin=3cm, rmargin=3cm, tmargin=3cm, bmargin=3cm]{geometry}

\usepackage{amsmath, amsfonts, amssymb, graphicx, enumerate, verbatim, hyperref, color, lipsum, ifxetex, ifluatex, etoolbox, tikz, framed, libertine, listings, caption, subcaption, wrapfig, sectsty, pdfcomment}

\usetikzlibrary{arrows,automata}

\newcommand{\HRule}{\rule{\linewidth}{1pt}}

\sectionfont{\LARGE}
\subsectionfont{\Large}
\subsubsectionfont{\large}

/home/tonussi/Git/maquinas-jflap/maquinas/color.tex

\hypersetup{colorlinks,citecolor=airforceblue, filecolor=ashgrey, linkcolor=amaranth, urlcolor=raspberrypink}

\lstset{language=Java, basicstyle=\ttfamily, keywordstyle=\color{javapurple}\bfseries, stringstyle=\color{javared}, commentstyle=\color{javagreen}, morecomment=[s][\color{javadocblue}]{/**}{*/}, numbers=left, numberstyle=\tiny\color{black}, stepnumber=1, numbersep=10pt, tabsize=2, showspaces=false, showstringspaces=false}


% conditional for xetex or luatex
\newif\ifxetexorluatex
\ifxetex
  \xetexorluatextrue
\else
  \ifluatex
    \xetexorluatextrue
  \else
    \xetexorluatexfalse
  \fi
\fi
%
\ifxetexorluatex%
  \usepackage{fontspec}
  \usepackage{libertine} % or use \setmainfont to choose any font on your system
  \newfontfamily\quotefont[Ligatures=TeX]{Linux Libertine O} % selects Libertine as the quote font
\else
  \usepackage[utf8]{inputenc}
  \usepackage[T1]{fontenc}
  \usepackage{libertine} % or any other font package
  \newcommand*\quotefont{\fontfamily{LinuxLibertineT-LF}} % selects Libertine as the quote font
\fi

\newcommand*\quotesize{60} % if quote size changes, need a way to make shifts relative
% Make commands for the quotes
\newcommand*{\openquote}
   {\tikz[remember picture,overlay,xshift=-4ex,yshift=-2.5ex]
   \node (OQ) {\quotefont\fontsize{\quotesize}{\quotesize}\selectfont``};\kern0pt}

\newcommand*{\closequote}[1]
  {\tikz[remember picture,overlay,xshift=4ex,yshift={#1}]
   \node (CQ) {\quotefont\fontsize{\quotesize}{\quotesize}\selectfont''};}

% select a colour for the shading
\colorlet{shadecolor}{Azure}

\newcommand*\shadedauthorformat{\emph} % define format for the author argument

% Now a command to allow left, right and centre alignment of the author
\newcommand*\authoralign[1] {
  \if#1l
    \def\authorfill{}\def\quotefill{\hfill}
  \else
    \if#1r
      \def\authorfill{\hfill}\def\quotefill{}
    \else
      \if#1c
        \gdef\authorfill{\hfill}\def\quotefill{\hfill}
      \else\typeout{Invalid option}
      \fi
    \fi
  \fi
}


% wrap everything in its own environment which takes one argument (author) and one optional argument
% specifying the alignment [l, r or c]
\newenvironment{shadequote}[2][l]%
{\authoralign{#1}
\ifblank{#2}
   {\def\shadequoteauthor{}\def\yshift{-2ex}\def\quotefill{\hfill}}
   {\def\shadequoteauthor{\par\authorfill\shadedauthorformat{#2}}\def\yshift{2ex}}
\begin{snugshade}\begin{quote}\openquote}
{\shadequoteauthor\quotefill\closequote{\yshift}\end{quote}\end{snugshade}}


\begin{document}
\begin{titlepage}
\begin{center}

\textsc{\large Universidade Federal de Santa Catarina}\\[1cm]

\includegraphics[width=0.3\textwidth]{ufsc}\\[1.5cm]

\textsc{\LARGE \bfseries Teoria da Computação \\ [0.8cm]}

\textsc{\LARGE \bfseries Problema da Correspondência de Post \\ [3cm]}


\begin{Large}
\textbf{Estudante}:
\begin{tabular}{|c}
Luc$\lambda$s Tonussi 12106577\\
\end{tabular} \\[0.5cm]
\end{Large}

\vfill

\begin{Large}
\textbf{Professora}:
\begin{tabular}{|c}
Jerusa Marchi \\
\end{tabular} \\[0.5cm]
\end{Large}

\vfill


\textbf{\today}

\end{center}
\end{titlepage}


\tableofcontents
\clearpage
\listoffigures
\clearpage


% Requisito mínimo: aplicação funcionando.
% As seguintes questões serão analisadas:

\section{Máquinas}

\qquad A seguir ser\~{a}o apresentadas as m\'{a}quinas para esse trabalho. Cada sess\~{a}o cont\'{e}m o nome do tipo de m\'{a}quina para que fique melhor organizado. Dentro de cada sess\~{a}o estão resumidamente contidas as informaç\~{o}es e simulaç\~{o}es sobre as m\'{a}quinas. Cada m\'{a}quina aborda uma linguagem de entrada diferente. As simulaç\~{o}es foram feitas usando jFlap \footnote{\href{http://www.cs.duke.edu/csed/jflap/}{jFlap}}. Antemão eu já aviso que eu infelizmente não achei solução para a última questão usando 2 Fitas, ou 3 Fitas. Eu bolei uma tática usando a mesma idéia da máquina anterior w\#w para construir uma ww (ou seja, se não existe \# como separador, coloca um separador e testa bit a bit). \pagebreak

\pagebreak
\subsection{Automato Finito}

$$ L(M)= \lbrace w \vert w \in \Sigma =  {\lbrace a,b,c \rbrace}^* \text{ e $w$ possui número par de $b$'s e número de $a$'s $+$ $c$'s é ímpar } \rbrace$$

\begin{figure}[ht]
  \centering
  \includegraphics[width=0.7\textwidth]{automato-finito} \flushright $\blacksquare$
  \caption{Automato Finito - Que aceita a linguagem regular definida acima.}
\end{figure}

\qquad Definição Formal
A coleção de linguagens regulares sobre um alfabeto $\Sigma$ é definida recursivamente seguindo as regras abaixo:

\begin{itemize}
  \item ($\emptyset$) é uma linguagem regular.
  \item $\lbrace \epsilon \rbrace$ é uma linguagem regular.
  \item Se $ a \in \Sigma$, então linguagem do conjunto unitário a é uma linguagem regular.
  \item Se $A \wedge B$ são linguagens regulares, então $A \cup B$, $A \bullet B$, e ${A}^*$ são linguagens regulares.
  \item Nenhuma outra linguagem sobre $\Sigma$ é regular. $\blacksquare$
\end{itemize}

\begin{figure}[ht]
  \centering
  \includegraphics[width=0.7\textwidth]{chomsky}
  \caption{Hierarquia de Chomsky, para mostrar onde estão localizadas as linguagens (ou gramáticas) regulares.}
\end{figure}


\qquad Uma tabela com alguns testes para esse automato finito. Como você pode ver esse automato só aceita número par de b's \textbf{e} número ímpar da soma de a's com c's.

\begin{tabular}{|c|c|}
  \hline $\emptyset$ & Rejected \\
  \hline bbb         & Rejected \\
  \hline bb          & Rejected \\
  \hline bbbb        & Rejected \\
  \hline bbbbb       & Rejected \\
  \hline aaaccc      & Rejected \\
  \hline aacc        & Rejected \\
  \hline ccaaaa      & Rejected \\
  \hline bbaacc      & Rejected \\
  \hline bbaaaaccc   & Accepted \\
  \hline bbaaaaaccc  & Rejected \\
  \hline ccaaaa      & Accepted \\
  \hline
\end{tabular}

\pagebreak
\subsection{Automato de Pilha}

$$ L(M)= \lbrace w \vert w \in \Sigma =  {\lbrace a,b,c \rbrace}^* \text{ e } w = {{a}^m}{{b}^n}{{c}^k} \text{ para } m \leqslant n  \leqslant 2m \text{ e } k \geqslant 0 \rbrace $$

\qquad O algoritmo basicamente é o seguinte: Para cada a guarda na pilha dois a's caso entre b's pode-se desempilhar no minimo o mesmo numero de a's que entraram na pilha ai entra o nao determinismo, onde vc pode empilhar dois a's ou empilhar apenas um a, quando b entrar desempilha os a's quando Z (fim de pilha no jflap) aparecer, você é livre para colocar quantos c's quiser e então eventualmente encerrar desempilhando Z.

\begin{figure}[ht]
  \centering
  \includegraphics[width=0.6\textwidth]{automato-pilha-nao-determinismos}
  \caption{Automato com Pilha - Os não-determinísmos marcados nesse Automato de Pilha.}
\end{figure}

\qquad Quando acontece o erro de b's menores que as. Observe que ele para em dois quadros de decisão. No automato de pilha não deterministico isso acontece bastante pois o sistema calcula muitas configurações possíveis.

\begin{figure}[ht]
  \centering
  \includegraphics[width=0.6\textwidth]{automato-pilha-bmenoresqueas-erro}
  \caption{Automato com Pilha - Erro - B's Menores que A's}
\end{figure}


\begin{figure}[ht]
  \centering
  \includegraphics[width=\textwidth]{automato-pilha} \flushright $\blacksquare$
  \caption{Automato com Pilha Não Determinístico - Que aceita a linguagem livre de contexto definida acima.}
\end{figure}

\begin{tabular}{|c|c|}
  \hline $\emptyset$           & Rejected \\
  \hline Z	                   & Reject \\
  \hline abc	                 & Accept \\
  \hline aabbcc	               & Accept \\
  \hline aabbbccc	             & Accept \\
  \hline aaabbbccc	           & Accept \\
  \hline aaaaaabbbbbbccccccc	 & Accept \\
  \hline aaaaaaabbbccccccccccc & Reject \\
  \hline aaaaaaaa	             & Reject \\
  \hline bbbbbbbb	             & Reject \\
  \hline ccccccccc	           & Reject \\
  \hline aaabbbc	             & Accept \\
  \hline cccc	                 & Reject \\
  \hline aaabbcccc	           & Reject \\
  \hline abbb	                 & Reject \\
  \hline aaaabbbbbbbbbcccc	   & Reject \\
  \hline abc	                 & Accept \\
  \hline
\end{tabular}

\pagebreak
\subsection{(TM) $M_1$ - Máquina de Turing Uma Fita}

$$ L(M)= \lbrace w \# w \vert w \in  {\lbrace 0,1 \rbrace}^* \rbrace $$

\qquad O algoritmo é começar da esquerda para direita procurando 0 ou 1 se 0 então agora eu procuro o separador \# Supostamente eu posso ter "x bits já marcados". Mas o objetivo é marcar igualmente o 0 ou o 1 que eu marquei do lado esquerdo. Uma vez que eu fiz isso, eu volto e faço novamente, até que eu só tenha x...\#x... Se isso acontecer a máquina aceita caso contrário ela rejeita.


\begin{tabular}{|c|c|}
  \hline $\lambda$              & Reject \\
  \hline \#                     & Accept \\
  \hline 0\#0                   & Accept \\
  \hline 1\#1                   & Accept \\
  \hline 00\#00                 & Accept \\
  \hline 11\#11                 & Accept \\
  \hline 11\#00                 & Reject \\
  \hline 00\#11                 & Reject \\
  \hline 10101010\#10101010     & Accept \\
  \hline 01010101\#01010101     & Accept \\
  \hline 1100\#1100             & Accept \\
  \hline 0011\#0011             & Accept \\
  \hline 0000011111\#0000011111 & Accept \\
  \hline
\end{tabular}

\begin{figure}[ht]
\centering
\includegraphics[width=0.7\textwidth]{turing-whashtagw} \flushright $\blacksquare$
\caption{Máquina de Turing Fita Única.}
\end{figure}

\pagebreak
\subsection{(TM) $M_1$ - Máquina de Turing Uma Fita 2}

$$ L(M)= \lbrace ww \vert w \in  {\lbrace 0,1 \rbrace}^* \rbrace \rightarrow MT \small (M_{n}) $$

\qquad A tática é procurar por um primeiro a ou um primeiro b eventual. Ao achar este primeiro item marca com um \# (hashtag). Agora percorre até o eventualmente acharmos o final da fita, até acharmos símbolo representando o espaço em branco. No JFlap é o símbolo de um quadrado em branco $\square$. Uma vez que achamos o quadrado em branco, agora voltamos e verificamos se a palavra de trás para frente é a mesma.

\begin{figure}[ht]
\centering
\includegraphics[width=\textwidth]{turing-ww} \flushright $\blacksquare$
\caption{Máquina de Turing (Deterministica).}
\end{figure}


\subsubsection{A idéia geral.}
\qquad O algoritmo esclarecido a seguir. (Pseudo-código, Nota: Coloquei alguns comandos por causa do listings do latex, Outra coisa é: Siga a identação do pseudo código.). A idéia é basicamente em um primeiro caso verificar se ww é par, e colocar uma marca forçada separando as palavras pela metade para se poder fazer um "and" bitwise. Se em um primeiro momento for ímpar o número de bits não tem como acontecer ww, então rejeita. Mas... se for par então vai para um segundo caso e aceita ou não dependendo se o "and" passar por todos os bits dando true.

\pagebreak
\begin{lstlisting}
while 0 or 1
  Troca 0 por X or 1 por Y & Mova Direita
  while 0 or 1 Mova Direita
  Mova Esquerda
  Troca 0 por X or 1 por Y & Mova Esquerda
  while 0 or 1 Mova Esquerda
  if X or Y Mova Direita

// length(w) % 2 != 0 ? Rejeita : Continua

if Branco Encerra
if X or Y Mova Esquerda
while X or Y Troca por a or 1 & Mova Esquerda
if Branco Mova Direita

while true
  if 0
    Troca 0 por X & Mova Direita
    while 0 or 1 or Branco Mova Direita
    Troca X por Branco & Mova Esquerda
    // Refeita Y (Paridade Falha)

    while 0 or 1 or Branco Mova Esquerda
    if X or Y Mova Direita
  else if 1
    Troca 1 por Y & Mova Direita
    while 0 or 1 or Branco Mova Direita
    Troca Y por Branco & Mova Esquerda

    // Rejeita X (Paridade Falha)
    while 0 or 1 or Branco Mova Esquerda
    if X or Y Mova Direita
if Branco Encerra
\end{lstlisting}

\begin{tabular}{|c|c|}
  \hline $\lambda$  & Reject \\
  \hline 01010101   & Accept \\
  \hline 000010000  & Reject \\
  \hline 111111111  & Reject \\
  \hline 000000000  & Reject \\
  \hline 11111111   & Accept \\
  \hline 00000000   & Accept \\
  \hline 10101010   & Accept \\
  \hline
\end{tabular}


\pagebreak
\section{Bibliografia}

[1] Introduction to the Theory of Computation (2nd Ed) by Michael Sipser, Thomson Course Technology © 2006, ISBN-13: 978-0-534-95097-2, ISBN-10: 0-534-95097-3

\end{document}
